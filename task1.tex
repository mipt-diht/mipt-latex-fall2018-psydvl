\documentclass[12pt]{article}

\usepackage[utf8]{inputenc}
\usepackage[russian]{babel}

\title{Домашняя работа №1}
\author{Попов Дмитрий}
\date{}

\begin{document}
	\maketitle
    \begin{flushright}
        \itshape{Audi multa,}\\
        \itshape{loquere pauce}
    \end{flushright}
    \vspace{20pt}
    
    Это мой не первый документ в системе компьютерной верстки \LaTeX.
    
    \begin{center}
        \textsf{\huge{<<Ура!!!>>}}
    \end{center}

    А теперь формулы. \textsc{Формула}~--- краткое и точное словесное выражение, определение или же ряд математических величин, выраженный условными знаками.\\[15pt]
    \hspace*{28pt}{\bfseries\Large Термодинамика}
    
    Уравнение Менделеева-Клапейрона~--- уравнение состояния идеального газа, имеющее вид: $pV=\nu RT,$ где $p$~--- давление, $V$~--- объём, занимаемый газом, $T$~--- температура газа, $\nu$~--- количество вещества газа, а $R$~--- универсальная газовая постоянная.\\[15pt]
    \hspace*{28pt}{\bfseries\Large Геометрия \hfill Планиметрия}

    Для плоского треугольника со сторонами $a, b, c$ и с углом $\alpha,$ лежащим против стороны $a,$ справедливо соотношение:
    $$a^2=b^2+c^2-2bc\cos\alpha,$$
    из которого можно выразить косинус угла треугольника:
    $$\cos\alpha=\frac{a^2-b^2-c^2}{2bc}.$$
    \newpage % or \vfill\break
    Пусть $p$~--- полупериметр треугольника, тогда путем несложных преобразований можно получить, что
    $$\tg\frac\alpha2=\sqrt{\frac{(p-b)(p-c)}{p(p-a)}}.$$
    \vspace{1cm}
    На сегодня, пожалуй, хватит... Удачи!

\end{document}
